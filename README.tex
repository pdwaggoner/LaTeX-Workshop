\documentclass[11pt]{article}
\usepackage[latin1]{inputenc}
\usepackage{amsmath}
\usepackage{amsfonts}
\usepackage{amssymb}
\usepackage{graphicx}
\usepackage[left=1.00in, right=1.00in, top=1.00in, bottom=1.00in]{geometry}
\usepackage[english]{babel}

\usepackage{fancyhdr}

\pagestyle{fancy}
\fancyhf{}
\rhead{University of Houston, Political Science}
\lhead{Learning \LaTeX: \hspace{.01 cm} Overview}
\rfoot{Page \thepage}

\graphicspath{/Users/bpwaggo/Dropbox/Dissertation/Ch. 3 - Issue Specialist/Main Document}
%\usepackage{pdflscape} % USE THIS INSTEAD: (just after \begin{table}) \resizebox{\linewidth}{!}{

\usepackage[round]{natbib}
\bibliographystyle{apsr}

\begin{document}
	
	\title{Learning \LaTeX \hspace{.01 cm} Weekly Workshop Series \\
		\vspace{1cm}
	\large Outline of Topics Covered \\
		\vspace{1cm}}
	\author{Philip D. Waggoner\footnote{{\texttt{philip.waggoner@gmail.com}}. This document was prepared by Philip Waggoner for the \textit{Weekly Workshops on Learning \LaTeX}, hosted by the Deparment of Political Science, University of Houston.}}
	\date{ } % getting rid of the automatic date
	\maketitle


\clearpage

\section{Introductory Remarks}

	The workshops will meet most \textbf{Fridays} this semester at \textbf{12:00 pm} in \textbf{PGH 310}. We will meet anywhere from 1 -- 1.5 hours. The goal is to learn and retain for future use. Thus, too much information in a single session could limit the applicability of stuff we learn in this workshop. As such, it's important to reiterate that this workshop is intended to be a concise overview of \LaTeX and its basic features to set you up to be able to use it on your own down the line. But as such, this workshop series will merely scratch the surface. 
	
\section{Overview of Topics \& Tentative Schedule}

	*The goal is to have workshops mostly weekly. There are a few weeks, as noted below, where we will not have the workshops based on my own scheduling issues.

\begin{itemize}

	\item \textbf{Friday, September 22} -- Getting started, packages, typesetting, and basic document features (e.g., environments, essentials like punctuation/delimiters, backslashes/commands, commenting, typeface, etc.) \\
	
	\item \textbf{Friday, September 29} -- Graphics \& Tables (inserting and drawing) \\
	
	\item \textbf{Friday, October 13} -- Equations, Lists, \& Numbering (e.g., Greek letters, itemize, enumerate, centering, etc.) \\

	\item \textbf{Friday, October 20} -- Presentations \& Beamer, Week 1 \\

	\item \textbf{Friday, October 27} -- Presentations \& Beamer, Week 2 \\

	\item \textbf{Friday, November 3} -- Bibliographies \& References \\

	\item \textbf{Friday, November 17} \textit{(Final Week)} -- Review, any extra topics not covered, outstanding questions/concerns, and moving forward

\end{itemize}

\end{document}
